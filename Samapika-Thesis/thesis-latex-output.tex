% LaTeX Document for Samapika Thesis - Reduced to Canonical Conversion Differences
% Generated for publication-ready inclusion of visualizations and data
%
% COMPILATION INSTRUCTIONS:
% This document uses longtable for tables that may span multiple pages.
% Compile with: pdflatex thesis-latex-output.tex
% Run twice to resolve all cross-references.
%
% IMPORTANT: All image paths are relative to the Samapika-Thesis directory.
% Make sure to compile from that directory or adjust \graphicspath accordingly.

\documentclass[11pt,a4paper]{article}

% Required packages
\usepackage{graphicx}
\usepackage{booktabs}
\usepackage{longtable}
\usepackage{multirow}
\usepackage{array}
\usepackage{caption}
\usepackage{subcaption}
\usepackage[margin=1in]{geometry}
\usepackage{hyperref}
\usepackage{xcolor}
\usepackage{float}  % For better float control

% Table column types
\newcolumntype{L}[1]{>{\raggedright\arraybackslash}p{#1}}
\newcolumntype{C}[1]{>{\centering\arraybackslash}p{#1}}
\newcolumntype{R}[1]{>{\raggedleft\arraybackslash}p{#1}}

% Float control - prevent figures from floating too far from their reference
\setcounter{topnumber}{3}
\setcounter{bottomnumber}{2}
\setcounter{totalnumber}{5}
\renewcommand{\topfraction}{0.85}
\renewcommand{\bottomfraction}{0.7}
\renewcommand{\textfraction}{0.15}
\renewcommand{\floatpagefraction}{0.66}

\title{Analysis of Linguistic Transformations in News Headlines: \\
From Reduced to Canonical Forms}
\author{Samapika Thesis}
\date{\today}

\begin{document}

\maketitle
\tableofcontents
\newpage

%==============================================================================
\section{Introduction}
%==============================================================================

This document presents a comprehensive analysis of linguistic transformations
that occur when converting reduced news headlines to their canonical forms.
The analysis covers three major Indian English newspapers: The Hindu (TH),
Hindustan Times (HT), and Times of India (ToI).

The analysis examines transformations across multiple linguistic dimensions:
\begin{itemize}
    \item \textbf{Dependency Relations:} Changes in syntactic dependencies
    \item \textbf{Constituency Structure:} Modifications to phrase structure
    \item \textbf{Lexical Features:} Word-level changes (POS, lemma, form)
    \item \textbf{Functional Elements:} Addition/deletion of function words
    \item \textbf{Structural Changes:} Clause type and constituent movement
\end{itemize}

%==============================================================================
\section{Feature Ontology}
%==============================================================================

\subsection{Overview}

This section presents the linguistic feature ontology used to classify and
analyze transformations between reduced (headline) and canonical forms. The
ontology is organized hierarchically into three main categories shown in
Table~\ref{tab:ontology-overview}.

\begin{table}[H]
\centering
\caption{Feature ontology categories for headline analysis}
\label{tab:ontology-overview}
\small
\begin{tabular}{@{}lll@{}}
\toprule
\textbf{Category} & \textbf{Dimension} & \textbf{Purpose} \\
\midrule
H-Struct & Headline Structure & Discourse organization \\
H-Type & Headline Type & Sentence completeness \\
F-Type & Fragment Type & Phrase-level structure \\
\bottomrule
\end{tabular}
\end{table}

\subsection{Complete Feature Ontology Schema}

Table~\ref{tab:ontology-full} presents the complete feature ontology with all
values, descriptions, and examples.

\begin{longtable}{@{}llL{4.5cm}L{5.5cm}@{}}
\caption{Complete feature ontology for headline transformations}
\label{tab:ontology-full} \\
\toprule
\textbf{Feature} & \textbf{Value} & \textbf{Description} & \textbf{Example} \\
\midrule
\endfirsthead

\multicolumn{4}{c}%
{\tablename\ \thetable\ -- Continued from previous page} \\
\toprule
\textbf{Feature} & \textbf{Value} & \textbf{Description} & \textbf{Example} \\
\midrule
\endhead

\midrule
\multicolumn{4}{r}{\textit{Continued on next page}} \\
\endfoot

\bottomrule
\endlastfoot

\multirow{2}{*}{\textbf{H-Struct}}
& sg-line & Headlines consisting of a single line, representing the most common headline structure. These typically show moderate structural complexity.
& Hospital issues special cards \\
\cmidrule{2-4}
& micro-disc & Headlines consisting of multiple clauses or sentences, representing more complex discourse structures that include questions, elaborations, or multiple propositions.
& I'm looking to retire in a warm place that has a 'socially liberal mindset' and lots of live music --- and I'm a die-hard skier. Where should I go? \\
\midrule

\multirow{2}{*}{\textbf{H-Type}}
& frag & Incomplete sentences lacking full clausal structure, typically missing subjects, verbs, or both. This is the dominant pattern in news headlines.
& Answers for Chakravyuh \\
\cmidrule{2-4}
& non-frag & Complete sentences with Subject+Verb+Object structure, more common in feature articles and opinion pieces.
& Full clausal structures with complete predication \\
\midrule

\multirow{2}{*}{\textbf{F-Type}}
& complex-compound & Fragments composed of compound noun phrases or multi-word expressions with internal complexity.
& Dark charm \\
\cmidrule{2-4}
& phrase & Phrasal fragments of various types (NP, VP, PP, etc.) that lack full clausal structure.
& A burning issue; At his best \\

\end{longtable}

\subsection{Feature Descriptions}

\subsubsection{Headline Structure (H-Struct)}

The \textbf{H-Struct} feature classifies headlines based on their discourse organization:

\begin{description}
    \item[Single Line (sg-line):] The standard headline format consisting of
    a single line or simple clause structure.

    \item[Micro-Discourse (micro-disc):] More complex headlines spanning multiple
    clauses or containing embedded questions, typical in feature journalism.
\end{description}

\subsubsection{Headline Type (H-Type)}

The \textbf{H-Type} feature distinguishes between complete and fragmentary structures:

\begin{description}
    \item[Fragment (frag):] The predominant headline style, lacking full
    predication. These headlines often omit function words, auxiliaries,
    and determiners.

    \item[Non-Fragment (non-frag):] Complete sentences with explicit subjects,
    finite verbs, and full argument structure. More common in opinion pieces
    and feature articles.
\end{description}

\subsubsection{Fragment Type (F-Type)}

The \textbf{F-Type} feature further categorizes fragmentary headlines:

\begin{description}
    \item[Complex Compound:] Noun-noun compounds or multi-word expressions
    functioning as unified concepts without additional syntactic structure.

    \item[Phrase:] Various phrasal categories (NP, PP, VP, etc.) that form
    incomplete but structured expressions.
\end{description}

%==============================================================================
\section{Global Analysis Across All Newspapers}
%==============================================================================

\subsection{Overall Feature Distribution}

Figure~\ref{fig:global-features} presents the global distribution of linguistic
features across all three newspapers, showing the relative frequency of different
transformation types.

\begin{figure}[htbp]
    \centering
    \includegraphics[width=0.95\textwidth]{Global/ALL-global_features.png}
    \caption{Global distribution of linguistic features across all newspapers.
    The visualization shows that constituent movement (CONST-MOV) and dependency
    relation changes (DEP-REL-CHG) are the dominant transformation types, together
    accounting for approximately 70\% of all transformations.}
    \label{fig:global-features}
\end{figure}

\begin{longtable}{@{}llrr@{}}
\caption{Global feature distribution across all newspapers}
\label{tab:global-features} \\
\toprule
\textbf{Feature ID} & \textbf{Feature Name} & \textbf{Count} & \textbf{Percentage} \\
\midrule
\endfirsthead

\multicolumn{4}{c}%
{\tablename\ \thetable\ -- Continued from previous page} \\
\toprule
\textbf{Feature ID} & \textbf{Feature Name} & \textbf{Count} & \textbf{Percentage} \\
\midrule
\endhead

\midrule
\multicolumn{4}{r}{\textit{Continued on next page}} \\
\endfoot

\bottomrule
\endlastfoot

CONST-MOV & Constituent Movement & 11,485 & 37.23\% \\
DEP-REL-CHG & Dependency Relation Change & 9,892 & 32.07\% \\
CLAUSE-TYPE-CHG & Clause Type Change & 2,728 & 8.84\% \\
FW-DEL & Function Word Deletion & 2,241 & 7.26\% \\
TED & Tree Edit Distance & 1,034 & 3.35\% \\
LENGTH-CHG & Sentence Length Change & 1,022 & 3.31\% \\
C-DEL & Content Word Deletion & 720 & 2.33\% \\
C-ADD & Content Word Addition & 540 & 1.75\% \\
HEAD-CHG & Dependency Head Change & 282 & 0.91\% \\
CONST-REM & Constituent Removal & 254 & 0.82\% \\
FW-ADD & Function Word Addition & 170 & 0.55\% \\
FEAT-CHG & Morphological Feature Change & 129 & 0.42\% \\
CONST-ADD & Constituent Addition & 124 & 0.40\% \\
FORM-CHG & Surface Form Change & 89 & 0.29\% \\
POS-CHG & Part of Speech Change & 89 & 0.29\% \\
LEMMA-CHG & Lemma Change & 22 & 0.07\% \\
VERB-FORM-CHG & Verb Form Change & 14 & 0.05\% \\
TOKEN-REORDER & Token Reordering & 12 & 0.04\% \\
\end{longtable}

\subsection{Parse Type Comparison}

Figure~\ref{fig:parse-type-comparison} compares the distribution of features
across dependency and constituency parse representations.

\begin{figure}[htbp]
    \centering
    \includegraphics[width=0.95\textwidth]{Global/ALL-parse_type_comparison.png}
    \caption{Comparison of feature distributions across dependency and constituency
    parse types. Different transformation types show distinct patterns: some features
    are exclusive to dependency parsing (e.g., DEP-REL-CHG, HEAD-CHG), while others
    are specific to constituency parsing (e.g., CONST-MOV, CLAUSE-TYPE-CHG).}
    \label{fig:parse-type-comparison}
\end{figure}

\subsection{Top Features Analysis}

Figure~\ref{fig:top-features} provides a detailed analysis of the most frequent
transformation features and their characteristics.

\begin{figure}[htbp]
    \centering
    \includegraphics[width=0.95\textwidth]{Global/ALL-top_features_analysis.png}
    \caption{Detailed analysis of top transformation features, including frequency
    distributions, concentration ratios, and transformation diversity metrics.}
    \label{fig:top-features}
\end{figure}

\subsection{Value Diversity Analysis}

Figure~\ref{fig:global-diversity} illustrates the diversity of transformation
values for each feature type.

\begin{figure}[htbp]
    \centering
    \includegraphics[width=0.95\textwidth]{Global/ALL-value_diversity_analysis.png}
    \caption{Value diversity analysis showing the variety of specific transformations
    within each feature category. Higher diversity indicates more varied transformation
    patterns for that feature.}
    \label{fig:global-diversity}
\end{figure}

%==============================================================================
\section{The Hindu (TH) Analysis}
%==============================================================================

\subsection{Feature Distribution}

Figure~\ref{fig:th-global} shows the distribution of linguistic features
specifically for The Hindu newspaper.

\begin{figure}[htbp]
    \centering
    \includegraphics[width=0.95\textwidth]{TH/TH-global_features.png}
    \caption{Distribution of linguistic transformation features in The Hindu.}
    \label{fig:th-global}
\end{figure}

\subsection{Cross-Dimensional Analysis}

Figure~\ref{fig:th-cross} presents the cross-dimensional analysis showing
relationships between different transformation types.

\begin{figure}[htbp]
    \centering
    \includegraphics[width=0.95\textwidth]{TH/TH-cross_dimensional_analysis.png}
    \caption{Cross-dimensional analysis for The Hindu, showing correlations and
    patterns across different linguistic dimensions.}
    \label{fig:th-cross}
\end{figure}

\subsection{Feature Coverage Heatmap}

Figure~\ref{fig:th-heatmap} displays the feature coverage across different
parse types and contexts.

\begin{figure}[htbp]
    \centering
    \includegraphics[width=0.95\textwidth]{TH/TH-feature_coverage_heatmap.png}
    \caption{Feature coverage heatmap for The Hindu showing the intensity and
    distribution of different transformation types across parse types.}
    \label{fig:th-heatmap}
\end{figure}

\subsection{Value Diversity}

Figure~\ref{fig:th-diversity} shows the diversity of transformation values
in The Hindu dataset.

\begin{figure}[htbp]
    \centering
    \includegraphics[width=0.95\textwidth]{TH/TH-value_diversity_analysis.png}
    \caption{Value diversity analysis for The Hindu newspaper.}
    \label{fig:th-diversity}
\end{figure}

\subsection{Detailed Statistics}

Table~\ref{tab:th-cross-analysis} presents a comprehensive breakdown of all
transformation features observed in The Hindu corpus, separated by parse type
(dependency vs. constituency).

\begin{longtable}{@{}llllrr@{}}
\caption{The Hindu: Cross-dimensional analysis by newspaper and parse type}
\label{tab:th-cross-analysis} \\
\toprule
\textbf{Newspaper} & \textbf{Parse Type} & \textbf{Feature ID} & \textbf{Feature Name} & \textbf{Count} & \textbf{\%} \\
\midrule
\endfirsthead

\multicolumn{6}{c}%
{\tablename\ \thetable\ -- Continued from previous page} \\
\toprule
\textbf{Newspaper} & \textbf{Parse Type} & \textbf{Feature ID} & \textbf{Feature Name} & \textbf{Count} & \textbf{\%} \\
\midrule
\endhead

\midrule
\multicolumn{6}{r}{\textit{Continued on next page}} \\
\endfoot

\bottomrule
\endlastfoot

The-Hindu & dependency & FW-DEL & Function Word Deletion & 1,661 & 17.79\% \\
The-Hindu & dependency & C-ADD & Content Word Addition & 200 & 2.14\% \\
The-Hindu & dependency & POS-CHG & Part of Speech Change & 83 & 0.89\% \\
The-Hindu & dependency & FORM-CHG & Surface Form Change & 158 & 1.69\% \\
The-Hindu & dependency & DEP-REL-CHG & Dependency Relation Change & 5,284 & 56.60\% \\
The-Hindu & dependency & FEAT-CHG & Morphological Feature Change & 95 & 1.02\% \\
The-Hindu & dependency & LENGTH-CHG & Sentence Length Change & 1,117 & 11.97\% \\
The-Hindu & dependency & C-DEL & Content Word Deletion & 447 & 4.79\% \\
The-Hindu & dependency & HEAD-CHG & Dependency Head Change & 198 & 2.12\% \\
The-Hindu & dependency & FW-ADD & Function Word Addition & 49 & 0.52\% \\
The-Hindu & dependency & VERB-FORM-CHG & Verb Form Change & 11 & 0.12\% \\
The-Hindu & dependency & TOKEN-REORDER & Token Reordering & 6 & 0.06\% \\
The-Hindu & dependency & LEMMA-CHG & Lemma Change & 26 & 0.28\% \\
\midrule
The-Hindu & constituency & CONST-MOV & Constituent Movement & 5,705 & 66.09\% \\
The-Hindu & constituency & CLAUSE-TYPE-CHG & Clause Type Change & 1,500 & 17.38\% \\
The-Hindu & constituency & TED & Tree Edit Distance & 1,058 & 12.26\% \\
The-Hindu & constituency & CONST-ADD & Constituent Addition & 91 & 1.05\% \\
The-Hindu & constituency & CONST-REM & Constituent Removal & 278 & 3.22\% \\
\end{longtable}

%==============================================================================
\section{Hindustan Times (HT) Analysis}
%==============================================================================

\subsection{Feature Distribution}

Figure~\ref{fig:ht-global} shows the distribution of linguistic features
for Hindustan Times.

\begin{figure}[htbp]
    \centering
    \includegraphics[width=0.95\textwidth]{HT/HT-global_features.png}
    \caption{Distribution of linguistic transformation features in Hindustan Times.}
    \label{fig:ht-global}
\end{figure}

\subsection{Cross-Dimensional Analysis}

Figure~\ref{fig:ht-cross} presents the cross-dimensional analysis for
Hindustan Times.

\begin{figure}[htbp]
    \centering
    \includegraphics[width=0.95\textwidth]{HT/HT-cross_dimensional_analysis.png}
    \caption{Cross-dimensional analysis for Hindustan Times showing patterns
    across different linguistic dimensions.}
    \label{fig:ht-cross}
\end{figure}

\subsection{Feature Coverage Heatmap}

Figure~\ref{fig:ht-heatmap} displays the feature coverage heatmap for
Hindustan Times.

\begin{figure}[htbp]
    \centering
    \includegraphics[width=0.95\textwidth]{HT/HT-feature_coverage_heatmap.png}
    \caption{Feature coverage heatmap for Hindustan Times.}
    \label{fig:ht-heatmap}
\end{figure}

\subsection{Value Diversity}

Figure~\ref{fig:ht-diversity} shows the diversity of transformation values
in Hindustan Times.

\begin{figure}[htbp]
    \centering
    \includegraphics[width=0.95\textwidth]{HT/HT-value_diversity_analysis.png}
    \caption{Value diversity analysis for Hindustan Times.}
    \label{fig:ht-diversity}
\end{figure}

%==============================================================================
\section{Times of India (ToI) Analysis}
%==============================================================================

\subsection{Feature Distribution}

Figure~\ref{fig:toi-global} shows the distribution of linguistic features
for Times of India.

\begin{figure}[htbp]
    \centering
    \includegraphics[width=0.95\textwidth]{ToI/ToI-global_features.png}
    \caption{Distribution of linguistic transformation features in Times of India.}
    \label{fig:toi-global}
\end{figure}

\subsection{Cross-Dimensional Analysis}

Figure~\ref{fig:toi-cross} presents the cross-dimensional analysis for
Times of India.

\begin{figure}[htbp]
    \centering
    \includegraphics[width=0.95\textwidth]{ToI/ToI-cross_dimensional_analysis.png}
    \caption{Cross-dimensional analysis for Times of India.}
    \label{fig:toi-cross}
\end{figure}

\subsection{Feature Coverage Heatmap}

Figure~\ref{fig:toi-heatmap} displays the feature coverage heatmap for
Times of India.

\begin{figure}[htbp]
    \centering
    \includegraphics[width=0.95\textwidth]{ToI/ToI-feature_coverage_heatmap.png}
    \caption{Feature coverage heatmap for Times of India.}
    \label{fig:toi-heatmap}
\end{figure}

\subsection{Value Diversity}

Figure~\ref{fig:toi-diversity} shows the diversity of transformation values
in Times of India.

\begin{figure}[htbp]
    \centering
    \includegraphics[width=0.95\textwidth]{ToI/ToI-value_diversity_analysis.png}
    \caption{Value diversity analysis for Times of India.}
    \label{fig:toi-diversity}
\end{figure}

%==============================================================================
\section{Comparative Analysis}
%==============================================================================

\subsection{Cross-Newspaper Comparison}

This section presents a comparative analysis of transformation patterns
across the three newspapers.

\subsubsection{Key Findings}

\begin{itemize}
    \item \textbf{Constituent Movement:} The most frequent transformation type
    across all newspapers (37.23\% globally), indicating structural reordering
    is a primary strategy in headline canonicalization.

    \item \textbf{Dependency Relation Changes:} The second most common transformation
    (32.07\% globally), showing significant syntactic restructuring during
    canonicalization.

    \item \textbf{Clause Type Changes:} Substantial at 8.84\%, reflecting the
    conversion of non-finite and fragmentary clauses to finite main clauses.

    \item \textbf{Function Word Operations:} Combined function word deletion
    (7.26\%) and addition (0.55\%) show the asymmetric nature of headline
    compression, with deletions far exceeding additions.

    \item \textbf{Lexical Stability:} Low rates of lemma change (0.07\%) and
    form change (0.29\%) indicate that lexical content remains relatively
    stable, with transformations primarily affecting structure.
\end{itemize}

\subsection{Parse Type Patterns}

\paragraph{Dependency-Specific Features:}
\begin{itemize}
    \item DEP-REL-CHG: Changes in grammatical relations
    \item HEAD-CHG: Modifications to dependency heads
    \item Various word-level features (POS, FORM, FEAT, LEMMA changes)
\end{itemize}

\paragraph{Constituency-Specific Features:}
\begin{itemize}
    \item CONST-MOV: Constituent movement/fronting
    \item CLAUSE-TYPE-CHG: Clause type modifications
    \item CONST-REM/CONST-ADD: Constituent removal and addition
    \item TED: Tree edit distance measures
\end{itemize}

%==============================================================================
\section{Feature-Value Transformations}
%==============================================================================

\subsection{Transformation Patterns}

The following sections detail specific transformation patterns observed
in the data, focusing on the linguistic characteristics of each transformation type.

\subsubsection{Function Word Deletion Patterns}

The most common function word deletions include:
\begin{itemize}
    \item \textbf{Article Deletion (ART-DEL):} Removal of articles (a, an, the)
    is extremely common, as articles are typically omitted in headline style.

    \item \textbf{Auxiliary Deletion (AUX-DEL):} Auxiliary verbs are frequently
    deleted in headlines, contributing to the telegraphic style.

    \item \textbf{Subordinating Conjunction Deletion (SCONJ-DEL):} Subordinators
    are often removed, converting complex sentences to simpler structures.
\end{itemize}

\subsubsection{Dependency Relation Changes}

Key patterns in dependency relation changes:
\begin{itemize}
    \item \textbf{nsubj $\rightarrow$ root:} Subjects becoming root verbs in
    nominalized headlines

    \item \textbf{det $\rightarrow$ compound:} Determiners reanalyzed as part
    of compound structures

    \item \textbf{case $\rightarrow$ obl/nmod:} Prepositions and their objects
    restructured as oblique or nominal modifiers
\end{itemize}

\subsubsection{Constituent Movement}

Constituent fronting (CONST-FRONT) is the dominant movement type, where
important constituents are moved to headline-initial position for emphasis
and information structure.

\subsubsection{Clause Type Changes}

Common clause type transformations:
\begin{itemize}
    \item \textbf{Participial $\rightarrow$ Finite (Part $\rightarrow$ Fin):}
    Participial clauses converted to finite clauses

    \item \textbf{Fragment $\rightarrow$ Complete:} Fragmentary structures
    expanded to complete clauses
\end{itemize}

%==============================================================================
\section{Methodology Notes}
%==============================================================================

\subsection{Data Collection}

The analysis is based on parallel corpus of reduced (headline) and canonical
(expanded) versions of news articles from three major Indian English newspapers:
\begin{itemize}
    \item The Hindu (TH)
    \item Hindustan Times (HT)
    \item Times of India (ToI)
\end{itemize}

\subsection{Linguistic Annotation}

Both reduced and canonical versions were annotated using:
\begin{itemize}
    \item \textbf{Dependency Parsing:} Universal Dependencies framework
    \item \textbf{Constituency Parsing:} Penn Treebank-style phrase structure
    \item \textbf{Morphological Analysis:} POS tags and morphological features
\end{itemize}

\subsection{Feature Extraction}

Transformations were identified by comparing:
\begin{itemize}
    \item Syntactic structures (dependency trees, constituency trees)
    \item Lexical content (words, lemmas, POS tags)
    \item Functional elements (auxiliaries, determiners, conjunctions)
    \item Structural properties (clause types, constituent order)
\end{itemize}

%==============================================================================
\section{Conclusion}
%==============================================================================

This analysis reveals systematic patterns in the linguistic transformations
that occur when converting reduced news headlines to their canonical forms.
The dominant transformation types---constituent movement and dependency relation
changes---indicate that headline canonicalization primarily involves structural
rather than lexical modifications.

Key observations:
\begin{enumerate}
    \item Headlines employ consistent compression strategies across newspapers,
    with structural omissions (articles, auxiliaries) and reordering being
    primary mechanisms.

    \item The transformation patterns show clear asymmetries: deletions far
    exceed additions, and fronting is much more common than other movement types.

    \item Different parse types capture complementary aspects of transformation:
    dependency parsing reveals relational changes, while constituency parsing
    highlights phrasal restructuring.

    \item Despite surface diversity, the transformation patterns show underlying
    systematicity that could inform both generation and understanding of
    headline language.
\end{enumerate}

%==============================================================================
\appendix
%==============================================================================

\section{Feature Abbreviations}

This section provides a complete reference of all feature abbreviations used
throughout the document, organized alphabetically.

\begin{longtable}{@{}llL{7cm}@{}}
\caption{Complete list of feature abbreviations and their meanings}
\label{tab:feature-abbrev} \\
\toprule
\textbf{Abbreviation} & \textbf{Full Name} & \textbf{Description} \\
\midrule
\endfirsthead

\multicolumn{3}{c}%
{\tablename\ \thetable\ -- Continued from previous page} \\
\toprule
\textbf{Abbreviation} & \textbf{Full Name} & \textbf{Description} \\
\midrule
\endhead

\midrule
\multicolumn{3}{r}{\textit{Continued on next page}} \\
\endfoot

\bottomrule
\endlastfoot

C-ADD & Content Word Addition & Addition of content words (nouns, verbs, adjectives, adverbs) during canonicalization \\
C-DEL & Content Word Deletion & Removal of content words in reduced forms \\
CLAUSE-TYPE-CHG & Clause Type Change & Changes in clause finiteness (e.g., participial to finite) \\
CONST-ADD & Constituent Addition & Addition of phrasal constituents in canonical form \\
CONST-MOV & Constituent Movement & Reordering of constituents, especially fronting operations \\
CONST-REM & Constituent Removal & Removal of phrasal constituents in headlines \\
DEP-REL-CHG & Dependency Relation Change & Changes in grammatical relations between words \\
FEAT-CHG & Morphological Feature Change & Changes in morphological features (tense, number, etc.) \\
FORM-CHG & Surface Form Change & Changes in word surface form (e.g., capitalization) \\
FW-ADD & Function Word Addition & Addition of function words (articles, auxiliaries, etc.) \\
FW-DEL & Function Word Deletion & Removal of function words in headlines \\
HEAD-CHG & Dependency Head Change & Changes in syntactic head assignment \\
LEMMA-CHG & Lemma Change & Changes in lexical base form \\
LENGTH-CHG & Sentence Length Change & Overall change in number of tokens \\
POS-CHG & Part of Speech Change & Changes in word category assignment \\
TED & Tree Edit Distance & Overall structural distance between trees \\
TOKEN-REORDER & Token Reordering & Changes in linear word order \\
VERB-FORM-CHG & Verb Form Change & Specific changes to verbal morphology \\
\end{longtable}

\section{Data Availability}

All data, code, and visualizations used in this analysis are available in
the project repository:
\url{https://github.com/bhaashik/ReducedToCanonicalConvDiff.git}

\end{document}
